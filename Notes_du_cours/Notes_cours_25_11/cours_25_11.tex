\documentclass[12pt]{article}
\usepackage[utf8]{inputenc}
\usepackage[french]{babel}
\usepackage{amsmath,amsthm,amsfonts,amssymb}
\usepackage{lmodern}
\usepackage[top=2.4cm,bottom=2.4cm,left=2cm,right=2cm]{geometry}
\usepackage{hyperref}
\usepackage{multicol}
\usepackage{enumitem}
\usepackage{listings}
\usepackage[dvipsnames]{xcolor}
\usepackage{tikz}

%\date{}
\title{{\bf  Génie logiciel} \\
	Notes du cours de 25/11  \\
	{\small L3 Informatique appliquée 2022-2023} \\
	{\it \small MABROUK Fayez}}
\begin{document}
	\maketitle
	\newpage
	\section{UML pour modéliser la structure}
	\subsection{Découvrir des objets}
	\begin{itemize}
		\item[* ] Les objets peuvent être représentés à différentes granularités.
		\item[* ] Il est important de choisir la granularité appropriée à l'objectif du diagramme:
		\begin{itemize}
			\item[* ] Diagrammes de cas d'utilisation : diagramme de haut niveau destiné à la discussion : objets à faible granularité.
			\item[* ] Diagrammes de classes réalisés pour la conception du programme : objets à granularité élevée.
		\end{itemize}
	\item[* ] Exemple : un ordinateur portable peut être vu comme.. :
	\begin{itemize}
		\item[* ] un... objet ordinateur portable (faible granularité).
		\item[* ] la composition d'un châssis, trackpad, clavier, écran, carte mère, CPU, RAM, ...
		(granularité élevée).
	\end{itemize}
\item[* ] Il est possible de découvrir des objets par :
\begin{itemize}
	\item[* ] dynamique : en regardant quel objet doit recevoir le message.
	\item[* ] données : en analysant la structure de l'objet.
	
\end{itemize}
\item[* ] Exemple : on peut trouver des objets dans un ordinateur portable :
\begin{itemize}
	\item[* ] dynamiquement : Je veux exécuter un programme : d'abord, je déplace mon curseur avec le trackpad jusqu'à la barre de recherche ; ensuite, je tape le nom du programme.
	la barre de recherche ; ensuite je tape le nom du programme avec le clavier ; puis le CPU
	exécute le programme...
	\item[* ] à travers des données : quand je regarde les spécifications de mon ordinateur portable, je peux voir qu'il a un intel i7
	avec 16Go de RAM, un clavier AZERTY, ...
\end{itemize}
	\end{itemize}
\subsection{Types de diagrammes}
\begin{itemize}
	\item[* ]  Diagrammes de structure:
	\begin{itemize}
		\item[* ] \textcolor{red}{Diagramme de classes, diagramme d'objets,} diagramme de composants, diagramme de structure composite,
		Diagramme de paquetage et Diagramme de déploiement
	\end{itemize}
\end{itemize}
\end{document}